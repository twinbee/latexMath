\documentclass[10pt,fullpage]{beamer}
\mode<presentation>


\usepackage{amsmath,amssymb,amsthm,amsfonts} % Typical maths resource packages



\title{ Mathematics in \LaTeX}

\author{Matthew Bennett \\
{\small \today }}

 \date{ }

\begin{document}
\maketitle

\begin{frame}{Math Mode}
Before you can enter math commands, you must be in ``math mode''.

To enter math commands in the same line as your text, such as $x^2 -
11 = 0$ use a single \$ and another \$ to exit math mode.

For a more formal look, use the \$\$ symbol to begin and end the
line offset math mode. This will offset the math symbols onto a
separate line and center it on the page. For instance $$\sin^2 x +
cos^2 x = 1$$ will break the text and automatically center.

Math text will automatically eliminate spaces and italicize all
text. If we type regular text in math mode, it will $probably not
look right.$ We can use the \ text\{\} command while in math mode to
generate normal text, such as $h \text{  (the hypoteneuse)} = a^2 +
b^2$.

\end{frame}

\begin{frame}{Subscripts and Superscripts}

To raise something to a power, prefix the power with the carat
command. x\^{}y produces $$x^y$$.

Similarly, to produce a subscript, prefix it with an underscore
command. x\_{}d produces $$x_d$$.

\end{frame}

\begin{frame}[fragile]
\frametitle{List of Greek Symbols}
\begin{verbatim}
\alpha               \theta               o \tau
 \beta                \vartheta           \pi                 \upsilon
 \gamma               \gamma              \varpi              \phi
 \delta               \kappa              \rho                \varphi
 \epsilon             \lambda             \varrho             \chi
 \varepsilon          \mu                 \sigma              \psi
 \zeta                \nu                 \varsigma           \omega
 \eta                 \xi
 \Gamma               \Lambda             \Sigma              \Psi
 \Delta               \Xi                 \Upsilon            \Omega
 \Theta               \Pi                 \Phi
\end{verbatim}

$$\alpha               \theta               o \tau
 \beta                \vartheta           \pi                 \upsilon
 \gamma               \gamma              \varpi              \phi
 \delta               \kappa              \rho                \varphi
 \epsilon             \lambda             \varrho             \chi
 \varepsilon          \mu                 \sigma              \psi
 \zeta                \nu                 \varsigma           \omega
 \eta                 \xi
 \Gamma               \Lambda             \Sigma              \Psi
 \Delta               \Xi                 \Upsilon            \Omega
 \Theta               \Pi                 \Phi
$$
\end{frame}

\begin{frame}[fragile]
\frametitle{List of Binary Operators}
\begin{verbatim}
 \pm                  \cap                \diamond                    \oplus
 \mp                  \cup                \bigtriangleup              \ominus
 \times               \uplus              \bigtriangledown            \otimes
 \div                 \sqcap              \triangleleft               \oslash
 \ast                 \sqcup              \triangleright              \odot
 \star                \vee                \lhd^b                    \bigcirc
 \circ                \wedge              \rhd^b                    \dagger
 \bullet              \setminus           \unlhd^b                  \ddagger
 \cdot                \wr                 \unrhd^b                  \amalg
 +                    -
\end{verbatim}

$$
\pm                  \cap                \diamond                    \oplus
 \mp                  \cup                \bigtriangleup              \ominus
 \times               \uplus              \bigtriangledown            \otimes
 \div                 \sqcap              \triangleleft               \oslash
 \ast                 \sqcup              \triangleright              \odot
 \star                \vee                \lhd^b                    \bigcirc
 \circ                \wedge              \rhd^b                    \dagger
 \bullet              \setminus           \unlhd^b                  \ddagger
 \cdot                \wr                 \unrhd^b                  \amalg
 +                    -
$$
\end{frame}

\begin{frame}[fragile]
\frametitle{List of Relational Symbols}
\begin{verbatim}
 \leq                 \geq                \equiv              \models
 \prec                \succ               \sim                \perp
 \preceq              \succeq             \simeq              \mid
 \ll                  \gg                 \asymp              \parallel
 \subset              \supset             \approx             \bowtie
 \subseteq            \supseteq           \cong               \Join$^b$
 \sqsubset$^b$        \sqsupset$^b$       \neq                \smile
 \sqsubseteq          \sqsupseteq         \doteq              \frown
 \in                  \ni                 \propto             =
 \vdash               \dashv              <                   >
 :
\end{verbatim}

$$
 \leq                 \geq                \equiv              \models
 \prec                \succ               \sim                \perp
 \preceq              \succeq             \simeq              \mid
 \ll                  \gg                 \asymp              \parallel
 \subset              \supset             \approx             \bowtie
 \subseteq            \supseteq           \cong
 \neq                \smile
 \sqsubseteq          \sqsupseteq         \doteq              \frown
 \in                  \ni                 \propto             =
 \vdash               \dashv              <                   >
 :
$$

\section{For more} http://www.math.harvard.edu/texman/node21.html
lists all symbols available.

\end{frame}

\begin{frame}{Equation environment} The latex equation environment
automatically enters math mode, numbers the equation, and centers
the text on the page. Here we have used the equation environment
with a forward reference to Einstein's famous equation.
\ref{rest_energy}

\begin{equation}
E = mc^2 \label{rest_energy}
\end{equation}

\end{frame}


\begin{frame}[fragile]
\frametitle{Align environment} The align environment automatically
enters math mode, numbers the equation, and centers the text on the
page, just like the equation environment. Additionally, multiple
lines may be aligned and numbered.

\begin{verbatim}
\begin{align}
x&=y & X&=Y & a&=b+c\\
x�&=y� & X�&=Y� & a�&=b\\
x+x�&=y+y� & X+X�&=Y+Y� & a�b&=c�b
\end{align}
\end{verbatim}

\begin{align}
x&=y & X&=Y & a&=b+c\\
x�&=y� & X�&=Y� & a�&=b\\
x+x�&=y+y� & X+X�&=Y+Y� & a�b&=c�b
\end{align}

\end{frame}

\begin{frame}[fragile]
\frametitle{Smart symbols}

Some symbols will increase or decrease their size to match the other
symbols associated with them.

\begin{verbatim}
c^2 = \sqrt{a^2 + b^2}
\end{verbatim}

\begin{equation}
c^2 = \sqrt{a^2 + b^2}
\end{equation}

\begin{verbatim}
\int{_a}{^b} \frac{\sqrt{\sum{x^2 + y^2 + z^z}}}{6}ds
\end{verbatim}

\begin{equation}
\int_a^b \frac{\sqrt{\sum{sx^2 + y^2s^2 + z^z}}}{6}ds
\end{equation}

\end{frame}

\begin{frame}[fragile]
\frametitle{Smart Parenthesis}

You can use smart parenthesis, a type of smart symbol, to
automatically size the parenthesis in math mode to other smart
symbols.
\begin{verbatim}
$$ \left(\sum a_i^2\right)^{1/2} $$
\end{verbatim}


$$ \left(\sum a_i^2\right)^{1/2} $$

You can also use \begin{verbatim} \left[, \right], \left\{,
\right\}.\end{verbatim}

\end{frame}

\begin{frame}[fragile]

\frametitle{Cases}

\begin{verbatim}
\delta_{ij} =
\begin{cases}
1& i=j},\\
0& \text{Everywhere else}.
\end{cases}
\end{verbatim}

$$
\delta_{ij}=\begin{cases}
1& i=j,\\
0& \text{Everywhere else}.
\end{cases}
$$

\end{frame}

\begin{frame}[fragile]
\frametitle{Mathematical Nightmares}

Multiple integrals can be produced with int, iint, iiint, or
idotsint.

\begin{verbatim}
\idotsint f(x_1, x_2, \ldots) dx_1\ dx_2\ldots
\end{verbatim}

$$\idotsint f(x_1, x_2, \ldots) dx_1\ dx_2\ldots$$

Continued Fractions can be done by using cfrac (similar to frac).

\begin{verbatim}
\cfrac{1}{\sqrt{2}+ \cfrac{1}{\sqrt{2}+ \cfrac{1}{\sqrt{2}+\dotsb
}}}
\end{verbatim}

$$\cfrac{1}{\sqrt{2}+
\cfrac{1}{\sqrt{2}+ \cfrac{1}{\sqrt{2}+\dotsb }}}$$

\end{frame}

\begin{frame}[fragile]
\frametitle{Custom commands}

\begin{verbatim}
\providecommand{\abs}[1]{\lvert#1\rvert}
\end{verbatim}

\providecommand{\abs}[1]{\lvert#1\rvert}

This provides an absolute value command abs that can be used in math
mode as follows to produce $\abs {x - 5}$

\begin{verbatim}
$\abs {x - 5}$
\end{verbatim}

\end{frame}

\begin{frame}[fragile]
\frametitle{Named Theorems, Propositions, and other stuff}

\newtheorem{prop}{Proposition}

First define the name for a theorem.

\begin{verbatim}
\newtheorem{prop}{Proposition}
\end{verbatim}

After the theorem has been named, use as follows:

\begin{verbatim}
\begin{prop}
If the sea is sweet, then I am the King of England.
\end{prop}
\end{verbatim}

Output is as follows:

\begin{prop}
If the sea is sweet, then I am the King of England.
\end{prop}



\end{frame}

\end{document}
